\documentclass[]{article}

\usepackage[utf8]{inputenc}
\usepackage[T1]{fontenc}
\usepackage[french]{babel}

\usepackage{hyperref}
\usepackage{nameref}

\usepackage{mathtools}
\usepackage{mathrsfs}

\usepackage{graphicx}
\usepackage{float}
\usepackage{lscape}
\usepackage{wrapfig}
\usepackage{rotating}

\usepackage{fullpage}

\usepackage{xcolor}
\usepackage{listings}
\usepackage{fixltx2e}

\definecolor{darkblue}{rgb}{0.0,0.0,0.6}
\definecolor{darkred}{rgb}{0.3,0.0,0.0}

\lstset{%
	breaklines=true,
	columns=fullflexible,
	showstringspaces=false
}

\lstdefinelanguage{XML}
{%
	morestring=[b]",
	morestring=[s]{>}{<},
	morestring=[s]{"}{"},
	moredelim=[s][\color{black}]{>}{<},
	stringstyle=\color{darkred},
	identifierstyle=\color{purple},
	keywordstyle=\color{darkblue},
	morekeywords={number, name, difficulty, maxTime, dependency, width, height, xPos, yPos, rotation, canBeMoved, canBeRotated}
}

\begin{document}

\title{Rapport d'analyse\\
	Projet de structures de données II\\
	\smallskip
	{\small Activité d'Apprentissage \textsf{S-INFO-020}}\\
}
\author{
Membres du groupe:\\
LABEEUW Dorian (\texttt{dorian.labeeuw@student.umons.ac.be}) \and JOERTZ Jonathan (\texttt{jonathan.joertz@student.umons.ac.be})}

\date{Année Académique 2017-2018\\
Bachelier en Sciences Informatiques Bloc 3\\ 
\vspace{1cm}
Faculté des Sciences, Université de Mons}

\maketitle

\medskip
\begin{center} \today \end{center}
	\begin{abstract}
		Ce \emph{rapport d'analyse} est rendu dans le cadre de l'AA \textsf{S-INFO-020} ``Projet de structures de données II", dispensé par le professeur \emph{Bruyère Véronique} en année académique 2017-2018.
	\end{abstract}

\newpage

\tableofcontents

\newpage

\section{Description des structures de données}\label{sec:mainDesc}
\subsection{Description de \mathscr{Q}}\label{sub:mainQDesc}
\subsubsection{Description}\label{subsub:QDesc}
\subsubsection{Exemple}\label{subsub:QEx}
\subsubsection{Ordre sous-jacent}\label{subsub:QOrd}
\subsection{Description de \mathscr{T}}\label{sub:mainTDesc}
\subsubsection{Description}\label{subsub:TDesc}
\subsubsection{Exemple}\label{subsub:TEx}
\subsubsection{Ordre sous-jacent}\label{subsub:TOrd}

\section{Explication des algorithmes présentés}\label{sec:algos}
\subsection{Explication de \emph{FindIntersections(S)}}\label{sub:findInter}
\subsection{Explication de \emph{HandleEventPoint(p)}}\label{sub:eventPoint}
\subsection{Explication de \emph{FindNewEvent(s\textsubscript{l}, s\textsubscript{r}, p)}}\label{sub:findNew}

\section{Gestion des cas particuliers}\label{sec:part}
\subsection{Gestion des segments horizontaux}\label{sub:horiz}
\subsection{Gestion des points d'intersection à au moins 3 segments}\label{sub:interSeg}

\section{Description des étapes du programme}\label{sec:prog}
%ATTENTION À MENTIONNER LES STRUCTURES UTILISÉES!!!

\end{document}
